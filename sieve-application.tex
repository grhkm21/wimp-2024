%! TEX root = main.tex

\begin{frame}\frametitle{Why is the result wrong?}
Some of you may know about the Prime Number Theorem, which states that \(\pi(N) \sim \frac{N}{\log N}\), meaning our derived asymptotic is not tight.
\pause \\[5px]

The reason lies in the unpredictable behaviour of \(\mu(d)\). Indeed, there are several results on the ``pseudorandomness'' of \(\mu(d)\) ~\cite{GreenTao12}. Even its prefix sum \(\sum_{d = 1}^N \mu(d)\) isn't well understood for a long time.
\pause \\[5px]

Instead, Viggo Brun suggested replacing \(\mu(d)\) with better-behaved functions \(\lambda_d\) that allow us to control the error term better.
\end{frame}

\begin{frame}\frametitle{\insertsubsection}
Before we finish off, let me note that sieve theory is \textit{much} more general than what I demonstrated.
\pause \\[5px]

For example, one can tackle the twin prime conjecture by replacing the sieving sets \(\mathcal{S}_d = d\mathbb{N}\) with \(\mathcal{S}_p \coloneqq \left\{n : p \mid n(n + 2)\right\}\). (Why?)
\pause \\[5px]

Even though we don't know how to prove the twin prime conjecture, the more general question of ``bounded gaps between primes'' has been successfully tackled using more advanced constructions of the sieve method. (I love James Maynard!)
\pause \\[5px]

The sieve method is also used to attack weakenings of the Goldbach conjecture. Brun obtained the first result in this direction, proving the so-called ``\(9 + 9\)'' theorem, which means every even integer can be written as the sum of two numbers with at most \(9\) prime divisors. The most recent breakthrough is the ``\(1 + 2\)'' theorem, achieved by Chen JingRun in 1966. Note that there are fundamental barriers that prevents the method from proving the full Goldbach's conjecture.

Here are some citations so they show up in the next slide: ~\cite{CojocaruMurty05} ~\cite{Greaves01} ~\cite{Liu22}
\end{frame}

\begin{frame}\frametitle{References}
\printbibliography
\end{frame}
