%! TEX root = main.tex

\begin{frame}\frametitle{\insertsubsection}
Sieve theory is \textit{much} more general than what I demonstrated.
\pause \\[5px]

For example, one can tackle the twin prime conjecture by replacing the sieving sets \(\mathcal{S}_d = d\mathbb{N}\) with \(\mathcal{S}_p \coloneqq \left\{n : p \mid n(n + 2)\right\}\). Combined with the Brun's sieve, one gets the result mentioned before, where
\begin{exampleblock}{Twin not-primes}
\[
  \#\{n \leq N : \omega(n), \omega(n + 2) \leq 7\} \gg \frac{N}{\log^2 N}
\]
\end{exampleblock}

The constant \(7\) appears because using notation before, we set \(z > N^{1 / 8}\), meaning only prime divisors \(> z > N^{1 / 8}\) remains, of which there can only be \(7\) of.
\pause \\[5px]

Even though we don't know how to prove the twin prime conjecture, the more general question of ``bounded gaps between primes'', which asks ``what is the smallest prime gap that appears infinitely often'', has been tackled using the GPY and Maynard sieves.
\end{frame}

\begin{frame}\frametitle{\insertsubsection}
Finally, let's talk about the Goldbach conjecture! Most progress on it comes in the form of ``\(m + n\) Theorems'':
\begin{exampleblock}{\(m + n\) Theorem}
  All even integers \(\geq 4\) can be written as the sum of an integer with at most \(m\) prime divisors, and another integer with at most \(n\) prime divisors.
\end{exampleblock}
\pause

\begin{itemize}
  \item Brun (1920): \(9 + 9\) Theorem
  \item Buchstab: \(5 + 5\) Theorem
  \item Buchstab, Selberg: \(4 + 4\) Theorem
  \item Kuhn's weighted sieve: \(2 + 3\) Theorem
  \item Using GRH: \(1 + 7\) Theorem
  \item Chen JingRun (1966 in Chinese, 1973 in English): \(1 + 2\) Theorem
\end{itemize}

As usual, these mysterious numbers come from the choice of the sieving parameter \(z\): if we sieve away multiples of primes \(\leq (z \geq N^{1 / 10})\) from \(\mathcal{A}\), then the remaining numbers will all have prime divisors at least \(N^{1 / 10}\), meaning there can at most be \(9\) divisors. This is how Brun's result is achieved.
\\

Here are some citations so they show up in the next slide: ~\cite{CojocaruMurty05} ~\cite{Greaves01} ~\cite{Liu22} ~\cite{Maynard19} ~\cite{GPY05}
\end{frame}

\begin{frame}
Chen JingRun's seminal paper's abstract:

\centering
``In this paper we shall prove that every sufficiently large even integer is a sum of a prime and a product of at most 2 primes. The method used is simple without any complicated numerical calculations.''
\end{frame}

\begin{frame}\frametitle{References}
\printbibliography
\end{frame}
